\documentclass{standalone}
\usepackage{graphicx}	
\usepackage{amssymb, amsmath}
\usepackage{color}

\usepackage{tikz}
\usetikzlibrary{math, arrows.meta, decorations.pathreplacing}
\usepackage{pgfmath}

\definecolor{light}{RGB}{220, 188, 188}
\definecolor{mid}{RGB}{185, 124, 124}
\definecolor{dark}{RGB}{143, 39, 39}
\definecolor{highlight}{RGB}{180, 31, 180}
\definecolor{gray10}{gray}{0.1}
\definecolor{gray20}{gray}{0.2}
\definecolor{gray30}{gray}{0.3}
\definecolor{gray40}{gray}{0.4}
\definecolor{gray60}{gray}{0.6}
\definecolor{gray70}{gray}{0.7}
\definecolor{gray80}{gray}{0.8}
\definecolor{gray90}{gray}{0.9}
\definecolor{gray95}{gray}{0.95}

\tikzmath{
  function normal(\x, \m, \s) {
    return exp(-0.5 * (\x - \m) * (\x - \m) / (\s * \s) ) / (2.506628274631001 * \s);
  };
  function density(\x) {
    return normal(\x, 0.5, 0.5);
  };
  function phi(\x) {
    return \x * \x * \x / 4;
  };
  function invphi(\y) {
    if \y > 0 then {
      return +exp(ln(+4 * \y) / 3);
    } else {
      return -exp(ln(-4 * \y) / 3);
    };
  };
  function goodpushdensity(\y) {
    \x = invphi(\y);
    return density(\x) * (4 / 3) / (\x * \x);
  };
  function badpushdensity(\y) {
    \x = invphi(\y);
    return density(\x);
  };
}

\begin{document}

\begin{tikzpicture}[scale=1]

  \pgfmathsetmacro{\ys}{0.5};

  \begin{scope}[shift={(0, 0)}]
    \draw[white] (-4.5, -3) rectangle (3.75, 2.5);
        
    \begin{scope}
      \clip (-3, -2) rectangle (3, 2);
      \draw[domain=-1:1, smooth, samples=200, variable=\y, dark, line width=1] 
        plot (3 * \y, { \ys * goodpushdensity(\y) - 2});
      \draw[domain=-1:1, smooth, samples=200, variable=\y, light, line width=1] 
        plot (3 * \y, { \ys * badpushdensity(\y) - 2});
    \end{scope}
    
    \node[dark] at (1.85, 0) { $p \left( f^{-1}(y) \right) \cdot | J(y) |^{-1}$ };
    \node[light] at (1.75, -1.5) { $p \left( f^{-1}(y) \right)$ };
     
    \draw [->, >=stealth, line width=1] (-3.00, -2.015) -- +(0, 4.25);
    \draw [->, >=stealth, line width=1] (-3.015, -2.00) -- +(6.25, 0);
    
    \node at (0, -2.5) { $y$ };
  \end{scope}

\end{tikzpicture}

\end{document}  